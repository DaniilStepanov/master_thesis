%%%%%%%%%%%%%%%%%%%%%%%%%%%%%%%%%%%%%%%%%%%%%%%%%%%%%%%%%%%%%%%%%%%%%%%%%%%%%%%%
\conclusion
%%%%%%%%%%%%%%%%%%%%%%%%%%%%%%%%%%%%%%%%%%%%%%%%%%%%%%%%%%%%%%%%%%%%%%%%%%%%%%%%
Локализация найденных ошибок --- одна из основных проблем методов обеспечения качества программного обеспечения. В настоящее время данный процесс производится вручную, что часто приводит к большим временным затратам. Для решения проблемы локализации найденных ошибок в данной работе был предложен метод автоматической программной редукции.

В работе выполнен анализ предметной области программной редукции, определены основные особенности существующих методов. В результате исследования имеющихся методик и инструментов, реализующих различные алгоритмы программной редукции, была показана целесообразность создания такой методики для языка программирования Kotlin.

Основой предложенной технологии является комбинирование существующих технологий программной редукции и специфичных для целевого языка программирования трансформаций над различными видами представления программы. В работе представлено описание всех применяемых технологий и трансформаций, которые запускаются в определенном порядке до того момента, пока тестовый пример не перестанет изменяться.

На базе технологии разработан прототип для редукции тестов, приводящих к сбоям компилятора языка Kotlin. Прототип был апробирован на нескольких программных проектах. Полученные результаты свидетельствуют о том, что предложенная технология является эффективной для решения задач программной редукции. На практике представленная технология может применяться для сокращения временных затрат на локализацию найденных ошибок, если причина их происхождения не очевидна. 

Дальнейшее развитие результатов может выполняться и в теоретическом, и в практическом плане. В теоретическом плане можно дополнять и усложнять трансформации. Возможно, что применение алгоритма динамического программного среза также улучшит результаты и скорость работы подхода. Также возможно применение технологии машинного обучения для формирования квази-оптимального набора и порядка запуска разработанных трансформаций. В практическом плане возможно усовершенствование кода для ускорения работы прототипа программной редукции. Также возможным улучшением прототипа может являться параллельный запуск различных трансформаций.
