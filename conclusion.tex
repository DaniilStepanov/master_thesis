%%%%%%%%%%%%%%%%%%%%%%%%%%%%%%%%%%%%%%%%%%%%%%%%%%%%%%%%%%%%%%%%%%%%%%%%%%%%%%%%
\conclusion
%%%%%%%%%%%%%%%%%%%%%%%%%%%%%%%%%%%%%%%%%%%%%%%%%%%%%%%%%%%%%%%%%%%%%%%%%%%%%%%%
Одной из основых проблем методов обеспечения качества программного обеспечения является локализация найденных ошибок. Данный процесс в настоящее производится вручную, что зачастую приводит к большим временным затратам. В данной работе был предложен подход к решению данной проблемы --- метод автоматической программной редукции.

В работе выполнен анализ предметной области программной редукции, определены основные особенности существующих методов. В результате исследования имеющихся методик и инструментов, реализующих анализируемые алгоритмы программной редукции, была показана целесообразность создания такой методики для языка программирования Kotlin.

Основой предложенной технологии является комбинирование существующих технологий программной редукции и специфичных трансформаций для целевого языка программирования над различными видами представления программы. В работе представлено описание всех применяемых технологий и трансформаций, которые запускаются в определенном порядке до того момента, как тестовый пример не перестанет изменяться. 

На базе технологии разработан прототип для редукции тестов, приводяших к ошибке компилятора языка Kotlin. Полученные результаты показывают целесообразность применения технологии для задач программной редукции. Описанная технология может применяться в процессе разработки программного обеспечения для сокращения временных затрат на локализацию найденных ошибок, если причина их происхождения не очевидна. 

Дальнейшее развитие результатов может выполняться и в теоретическом, и в практическом плане. В теоретическом плане можно дополнять и усложнять трансформации. Возможно применение алгоритма динамического программного среза также улучшит результаты и скорость работы подхода. Также возможно применение технологии машинного обучения для формирования наиболее подходящего набора и порядка запуска разработанных трансформаций. В практическом плане возможно усовершенствование кода для ускорения работы прототипа программной редукции. Также возможным улучшением модуля будет параллельный запуск различных трансформаций.


