
\keywords{%
  редукция программного кода,
  локализация программных ошибок,
  качество программного обеспечения,
  тестирование
}

\abstractcontent{
Одной из основных проблем обеспечения качества программного обеспечения является локализация найденных в нем ошибок. Этот процесс в настоящее время производится вручную, что зачастую приводит к большим временным затратам. В данной работе предлагается подход к решению этой проблемы --- метод автоматической программной редукции.

Метод разрабатывался для языка программирования Kotlin. Основой предложенного метода является комбинирование существующих технологий программной редукции и трансформаций для целевого языка программирования над различными видами представления программы. В работе описываются все применяемые методы и трансформации. 

На базе метода разработан прототип для редукции тестов, приводящих к сбоям компилятора языка Kotlin. Полученные результаты показывают целесообразность применения технологии для задач программной редукции. Описанная технология может использоваться в процессе разработки программного обеспечения для сокращения временных затрат на локализацию найденных ошибок, если причина их происхождения не очевидна. 
}

\keywordsen{
  code reduction,
  bug localization,
  software quality,
  software testing
}

\abstractcontenten{
  One of the biggest problem of software quality assurance is bug localization. This process is usually done manually, which often leads to significant time spendings. In this thesis we propose an approach to this problem via automatic software reduction.

  We target Kotlin programming language. The basis of our solution is a combination of existing software reduction techniques and Kotlin-specific transformations implemented over various program representation. This paper describes all the methods and transformations we used in our approach.
  
  We implemented a prototype aimed at reduction of tests which cause Kotlin compiler errors and crashes. The evaluation results show the usability of our approach to software reduction problems --- it may be used in your everyday software development to reduce the time spent on localizing errors, if their source is not obvious.
}
