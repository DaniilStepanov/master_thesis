\chapter{Постановка задачи}
Для решения проблемы редукции программ необходимо сформулировать цель, задать ограничения, определить критерии эффективности и выявить требования к разработке. В данном разделе производится обоснование актуальности разработки для языка программирования Kotlin\footnote{https://kotlinlang.org/}. Также производится постановка задачи разработки технологии редукции программ и прототипа, реализующего данную технологию.

\section{Выбор целевого языка программирования}
Технология редукции программ разрабатывается для языка программирования Kotlin, который разрабатывается компанией JetBrains. Kotlin компилируется в JVM байт-код~\cite{dahm1999byte}, JavaScript~\cite{flanagan2006javascript} или LLVM~\cite{lattner2004llvm}. Главными преимуществами языка являются простота и полная совместимость с Java. Полное описание языка приведено в документации\footnote{https://kotlinlang.org/docs/reference/}.

Популярность языка Kotlin можно оценить с помощью отчётов раз­личных компаний, работающих в сфере информационных технологий. В рейтинге компании Tiobe\footnote{https://www.tiobe.com/tiobe-index/}, которая предоставляет достаточно полную и достоверную информацию о популярности языков, Kotlin находится на 49-ом месте. На данный момент языком пользуется около миллиона человек. Также необходимо отметить, что Kotlin продвигается компанией Google как основной язык разработки под платформу Android. 

В настоящее время для языка программирования Kotlin не существует никаких средств автоматической редукции. Помимо редукции разрабатываемых программ, одним из основных применений предлагаемой технологии является редукция ошибок компилятора Kotlin. Компилятор языка разрабатывается сравнительно небольшое количество времени, поэтому часто при разработке программ возникают ошибки компилятора, локализация которых в больших проектах может занимать много времени. Также существует генератор случайных тестов, результаты работы которого нашли много сбоев в работе компилятора, но эти результаты содержат много нерелевантной ошибкам информации. Перед отправкой найденных ошибок разработчикам компилятора эти тесты необходимо уменьшить, что в настоящее время делается вручную. По перечисленным причинам Kotlin --- один из самых актуальных языков для применения технологии редукции.
\section{Постановка задачи разработки технологии редукции программ}
Задача разработки технологии редукции программ состоит в автоматическом упрощении программ на языке программирования Kotlin, приводящих к ошибке. Для начала необходимо определиться с набором трансформаций и доказать их необходимость. Каждая из выбранных трансформаций может работать с определенным видом представления программы: текстовым, синтаксическим деревом, графом потока управления~\cite{harrold2005representation} и т.д. Поэтому необходимо уметь строить необходимое представление программы, а также реализовать инструменты их редактирования. Также для проверки воспроизведения ошибки необходимо решить задачу автоматического запуска целевой программы и разбора результата её работы.

Разрабатываемая технология должа быть безопасной, то есть результирующая программа должна гарантированно приводить к той же ошибке, что и исходная. Также технология должна быть эффективной --- редукция программ должна проводиться заметно быстрее, чем вручную. 

\section{Постановка задачи разработки прототипа, реализующего технологию редукции программ}
Разработанный модуль должен обладать следующей базовой функциональностью:
\begin{itemize}
\item принимать на вход целевой проект на языке Kotlin и тестовый пример, приводящий к ошибке;
\item редуцировать тестовый пример и выдавать результат меньший или равный исходному;
\item гарантировать безопасность редукции.
\end{itemize}
Язык разработки --- Kotlin. Разработанный прототип позволит оценить качество разработанной технологии редукции программ и её применимость к реальным проектам.

\section{Резюме}
В данном разделе поставлена задача разработки технологии редукции программ, определены основные требования и критерии успешности. Предложены и обоснованы пути ее решения. Также поставлена задача разработки прототипа, реализующего технологию редукции программ для проверки работоспособности технологии.