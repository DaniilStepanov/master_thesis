%%%%%%%%%%%%%%%%%%%%%%%%%%%%%%%%%%%%%%%%%%%%%%%%%%%%%%%%%%%%%%%%%%%%%%%%%%%%%%%%
\chapter{Постановка задачи}
%%%%%%%%%%%%%%%%%%%%%%%%%%%%%%%%%%%%%%%%%%%%%%%%%%%%%%%%%%%%%%%%%%%%%%%%%%%%%%%%
Для решения проблемы редукции программ необходимо сформулировать цель, задать ограничения, определить критерии эффективности и выявить требования к разработке. В данном разделе производится обоснование актуальности разработки технологии программной редукции для языка Kotlin\footnote{https://kotlinlang.org/}. Также производится постановка задачи разработки метода редукции программ и прототипа, реализующего данный метод.

%%%%%%%%%%%%%%%%%%%%%%%%%%%%%%%%%%%%%%%%%%%%%%%%%%%%%%%%%%%%%%%%%%%%%%%%%%%%%%%%
\section{Выбор целевого языка программирования}
%%%%%%%%%%%%%%%%%%%%%%%%%%%%%%%%%%%%%%%%%%%%%%%%%%%%%%%%%%%%%%%%%%%%%%%%%%%%%%%%
Технология редукции программ будет создаваться для языка программирования Kotlin, разрабатываемого компанией JetBrains. Kotlin компилируется в JVM байт-код~\cite{dahm1999byte}, JavaScript~\cite{flanagan2006javascript} или LLVM~\cite{lattner2004llvm}. По сравнению с другими языками под JVM, преимуществами Kotlin являются простота и полная совместимость с Java. Полное описание языка приведено в документации\footnote{https://kotlinlang.org/docs/reference/}.

Язык Kotlin продвигается компанией Google как основной язык разработки под платформу Android. Популярность языка в настоящее время можно оценить с помощью отчётов различных компаний, работающих в сфере информационных технологий. В рейтинге компании Tiobe\footnote{https://www.tiobe.com/tiobe-index/}, которая предоставляет различную информацию о популярности языков, Kotlin находится на 49-ом месте.

Помимо редукции разрабатываемых программ, одним из основных применений предлагаемой технологии может являться редукция ошибок компилятора Kotlin. На языке Kotlin написано относительно небольшое количество кода\footnote{по сравнению с наиболее популярными языками программирования}, поэтому иногда при разработке программ возникают ошибки компилятора, локализация которых в больших проектах может занимать много времени. Кроме того, существует генератор случайных тестов, результаты работы которого нашли множество сбоев в работе компилятора, но эти результаты содержат много нерелевантной ошибкам информации. Перед отправкой найденных ошибок разработчикам компилятора эти тесты необходимо уменьшить, что в настоящее время делается вручную. По перечисленным причинам Kotlin --- один из самых актуальных языков для применения технологии редукции.

%%%%%%%%%%%%%%%%%%%%%%%%%%%%%%%%%%%%%%%%%%%%%%%%%%%%%%%%%%%%%%%%%%%%%%%%%%%%%%%%
\section{Постановка задачи разработки технологии редукции программ}
%%%%%%%%%%%%%%%%%%%%%%%%%%%%%%%%%%%%%%%%%%%%%%%%%%%%%%%%%%%%%%%%%%%%%%%%%%%%%%%%
Задача разработки технологии редукции состоит в создании метода автоматической локализации ошибок в программах на языке Kotlin. Для решения этой задачи сначала необходимо определиться с набором трансформаций и доказать их необходимость. Каждая из выбранных трансформаций может работать с определенным видом представления программы. Это может быть текстовое представление, синтаксическое дерево, граф потока управления~\cite{harrold2005representation} и~т.д. Поэтому необходимо уметь строить необходимое представление программы, а также реализовать инструменты его редактирования. Также для проверки воспроизведения ошибки необходимо решить задачу автоматического запуска целевой программы и разбора результата ее работы.

Кроме того, разрабатываемая технология должна быть безопасной, т.е. результирующая программа должна гарантированно приводить к той же ошибке, что и исходная. Кроме того, технология должна быть эффективной --- редукция программ должна проводиться быстрее, чем вручную.

%%%%%%%%%%%%%%%%%%%%%%%%%%%%%%%%%%%%%%%%%%%%%%%%%%%%%%%%%%%%%%%%%%%%%%%%%%%%%%%%
\section{Постановка задачи разработки прототипа, реализующего технологию редукции программ}
%%%%%%%%%%%%%%%%%%%%%%%%%%%%%%%%%%%%%%%%%%%%%%%%%%%%%%%%%%%%%%%%%%%%%%%%%%%%%%%%
Разработанный модуль должен обладать следующей базовой функциональностью:
\begin{itemize}
\item принимать на вход целевой проект на языке Kotlin и тестовый пример, приводящий к ошибке;
\item редуцировать тестовый пример и выдавать результат меньший или равный исходному;
\item гарантировать безопасность редукции.
\end{itemize}
Язык разработки --- Kotlin. Разработанный прототип позволит оценить качество предложенной технологии редукции программ и ее применимость к реальным проектам.

%%%%%%%%%%%%%%%%%%%%%%%%%%%%%%%%%%%%%%%%%%%%%%%%%%%%%%%%%%%%%%%%%%%%%%%%%%%%%%%%
\section{Резюме}
%%%%%%%%%%%%%%%%%%%%%%%%%%%%%%%%%%%%%%%%%%%%%%%%%%%%%%%%%%%%%%%%%%%%%%%%%%%%%%%%
В данном разделе поставлена задача разработки технологии редукции программ, определены основные требования и критерии успешности. Предложены и обоснованы пути ее решения. Также поставлена задача разработки прототипа, реализующего технологию редукции программ, для проверки работоспособности технологии.
