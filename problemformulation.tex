\chapter{Постановка задачи}
Для решения проблемы редукции программ необходимо сформулировать цель, задать ограничения, определить критерии эффективности и выявить требования к разработке. В данном разделе производится обоснование актуальности разработки для языка программирования Kotlin\footnote{https://kotlinlang.org/}. Также постановка задачи разработки технологии редукции программ и модуля, реализующего данную технологию.

\section{Выбор целевого языка программирования}
Технология редукции программ разрабатывалась для языка программирования Kotlin. Данный язык разрабатывается компанией JetBrains. Kotlin компилируется в JVM байт-код или в JavaScript. Главными преимуществами языка являются простота и полная совместимость с Java. Полное описание языка приведено в документации\footnote{https://kotlinlang.org/docs/reference/}.

Популярность языка Kotlin Пможно оценить с помощью отчётов раз­личных компаний, работающих в сфере информационных технологий. В рейтинге компании Tiobe\footnote{https://www.tiobe.com/tiobe-index/}, которая предоставляет достаточно полную и достоверную информацию о популярности языков, Kotlin находится на 49-ом месте. На данный момент языком пользуется около миллиона человек 
TODO ссылка на интервью?))). Также необходимо отметить, что Kotlin продвигается компанией Google как основной язык для разработки под платформу Android. 

Компилятор языка разрабатывается не очень большое время, поэтому содержит ошибки, которые часто выявляются при разработке программ. По этой же причине для него не существует инструментов редукции, поэтому все найденные ошибки необходимо локализовывать вручную. Поэтому Kotlin --- один из самых актуальных языков для применения технологии редукции.

TODO мб что-нибудь добавить о языке?
\section{Постановка задачи разработки технологии редукции программ}
Задача разработки технологии редукции программ состоит в автоматическом упрощении программ на языке программирования Kotlin, приводящих к ошибке. Для начала необходимо определиться с набором трансформаций и доказать их необходимость. Каждая из выбранных трансформаций может работать со своим видом представления программы: текстовым, деревом разбора, графом потока управления~\cite{harrold2005representation} и т.д. Поэтому необходимо уметь получать необходимое представление программы, а также реализовать инструменты их редактирования. Также для проверки воспроизведения ошибки необходимо решить задачу запуска целевой программы и разбора результата её работы.

Разрабатываемая технология должа быть точной, то есть результирующая программа должна гарантированно приводить к той же ошибке, что и исходная. Также технология должна быть эффективной --- редукция программ должна проводиться заметно быстрее, чем вручную. 


\section{Постановка задачи разработки прототипа, реализующего технологию редукции программ}
Разработанный модуль должен обладать следующей базовой функциональностью:
\begin{itemize}
\item принимать на вход целевой проект на языке Kotlin и тестовый пример, приводящий к ошибке;
\item редуцировать тестовый пример и выдавать результат меньший или равный исходному;
\item гарантировать безопасность редукции.
\end{itemize}
Язык разработки --- Kotlin. Разработанный прототип позволит оценить качество разработанной технологии редукции программ и её применимость к реальным проектам.

\section{Резюме}
В данном разделе поставлена задача разработки технологии редукции программ, определены основные требования и критерии успешности. Предложены и обоснованы пути решения задачи. Также поставлена задача разработки прототипа, реализующего технологию редукции программ для проверки работоспособности технологии.