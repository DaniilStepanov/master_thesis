\chapter{Разработка метода}
В соответствии с поставленной задачей, необходимо разработать метод редукции программ на языке Kotlin. Этот раздел посвящен разработке компонентов, выполняющих данную задачу. Основные идеи, положенные в основу компонентов, рассмотрены далее.

CompilatorCrushingTester
трансформации в цикле!!
Разрабатываемый метод программной редукции является гибрибдным и состоит из следующих этапов:
\begin{itemize}
	\item предварительное упрощение проекта;
	\item слайсинг;
	\item трансформации над PSI;
	\item трансормации над текстовым представлением программы;
	\item иерархический дельта-дебаггинг;
\end{itemize}
Для каждого этапа программной редукции в данном разделе описан алгоритм работы и обоснована его актуальность.

\section{Предварительное упрощение}
про функции и todo
\section{Слайсинг}
Слайсинг производится на следующих уровнях:
\begin{itemize}
	\item импортов;
	\item классов;
	\item функций;
	\item внутрипроцедурном;
\end{itemize}
\subsection{Слайсинг на уровне импортов}
Взрываются звездочки, строится дерево импортов, снизу вверх запускаются трансформации
\subsection{Слайсинг на уровне классов}
Смотрится, используется ли класс внутри другого
\subsection{Слайсинг на уровне функций}
Строится дерево и смотрится какая функция откуда вызывается
\subsection{Внутрипроцедурный слайсинг}
Стандратный слайсинг сзаду-наперёд
\section{Трансформации над PSI}
Опытным путем был образован необходимый набор трансформаций над PSI. Дерево представление в качестве редактируемого представления программы было выбрано ввиду сложности проведения данного типа трансформаций над текстовым представлением. Ниже приведено описание каждой трансформации. 

\textbf{Удаление комментарием}. Часто программа содержит большое количество комментариев, которые, очевидно, являются нерелевантной информацией по отношению к программной ошибке. Поэтому их необходимо удалять.

\textbf{Упрощение Элвис-оператора}. Система типов в языке программирования Kotlin нацелена на предотвращение ошибок, при которых происходит обращение к null значению. Поэтому система типов Kotlin отличает типы, которые могут иметь значение null(nullable), и которые не могут non-null). Если у нас есть nullable ссылка, благодаря элвис-оператору мы можем провести проверку этой ссылки на null и использовать ее в левой части оператора, либо использовать non-null значение в правой части оператора. И, если данный оператор не влияет на воспроизведение ошибки в программе, мы можем заменить его на правую часть. Пример?? 

\textbf{Удаление пустых управляющих инструкций}. Если какая-нибудь управляющая инструкция, например: if, for, try, when и т.д. не содержит тела и не влияет на воспроизведение ошибки, то можно ее удалить

\textbf{Удаление наследования}. При редукции программы, использующей ООП, часто представляется невозможным удалить информацию, нерелевантную к ошибке из-за каких-либо внутренних связей(наследование и т.д.). Но в некоторых случаях это наследование можно попытаться упростить. Данная трансформация рекурсивно проходит по всем классам-родителям класса, в котором содержится ошибка и удаляет поля с одинаковыми именами. TODO: плохо описал

\textbf{Удаление параметра из функции}. В результате применения различных методов редукции аргументы функции часто становятся неиспользуемыми. Для их удаления необходимо найти все вызовы данной функции и одновременно удалить параметр из функции и из всех ее вызовов. Необходимо учитывать, что в программе могут иметься другие функции с таким же названием и даже сигнатурой.

\textbf{Удаление значений из списка родительских классов}. В списке родительских классов, каждый класс, если он не влияет на возникновение программной ошибки, можно попытаться удалить, одновременно с этим необходимо удалить все перегружающиеся члены из удаляемого класса.

\textbf{Удаление неиспользуемых импортов и переводов строк}. При редукции часто многие импорты становятся неиспользуемыми. Их логично удалять. Также при редукции могут оставаться лишние переводы строк, усложняющие читаемость результирующего кода, которые также удаляются.

\textbf{Замена возвращаемого значения функции на константу}. Если функция имеет стандартный возвращаемый тип, например Int или String, то правую часть оператора перехода return можно заменить на константное значение в случае, если замена не будет влиять на воспроизведение ошибки. Например, если функция имеет возвращаемый тип Int, то можно заменить правую часть всех операторов return на 0.

\textbf{Замена типа возвращаемого значения функции на Unit}. Функция, не возвращающая никакого значения, в языке программирования Kotlin имеет тип Unit(аналог void в Java). Стоит отметить, что тип возвращаемого значения Unit может быть опущен. По аналогии с предыдущей трансформацией, если она не влияет на воспроизводимость ошибки, можно заменить тип возвращаемого значения на Unit и удалить все операторы перехода return.

\textbf{Упрощение конструктора класса}. В результате редукции часто параметры из конструктора класса становятся неиспользуемыми. В этом случае их можно удалить. Для этого необходимо удалить неиспользуемый оператор непосредственно из конструктора и из всех его вызовов.

\textbf{Упрощение цикла for}. Часто для воспроизведения ошибки цикл for становится нерелевантным, а тело цикла используется. В документации языка программирования Kotlin описано, что цикл for позволяет проходить по всем элементам объекта, имеющего итератор:
обладающего внутренней или внешней функцией iterator(), возвращаемый тип которой обладает внутренней или внешней функцией next(), и обладает внутренней или внешней функцией hasNext(), возвращающей Boolean. Поэтому в случае использования параметра цикла необходимо создать свойство, левая часть которого будет этим параметров, а правая --- вызов функций iterator() и next() у объекта, по которому производится итерация. Например для цикла {\ttfamily for (item in collection)} необходимо создать свойство {\ttfamily val item = collection.iterator().next()} и далее подставить его тело.

\textbf{Упрощение функций и свойств}. В случае не влияния на воспроизведение ошибки, все тела всех функций и правую часть всех свойств можно заменить на вызов специальной функции TODO() --- при обращении бросающая исключение UnsupportedOperationException.

\textbf{Упрощение оператора if}. Данная трансформация заменяет оператор if на какую-либо его ветку, в случае успеха после теста на воспроизведение ошибки. Необходимо различать в условии оператора сравнение переменной с каким-либо значением, например if (i == 0) и сравнение с каким-либо типом при помощи оператора is, например if (item is Type). Во втором случае возможно появление умных приведений, так как компилятор следит за is-проверками для неизменяемых значений и вставляет приведения автоматически, там, где они нужны. Поэтому необходимо вручную приводить переменную из условия к типу, с которым он сравнивается для true-ветки. Для этого необходимо использовать оператор as: item as Type.

\textbf{Упрощение наследования}. Данная трансформация направлена на упрощение наследования. TODO подробнее!

\textbf{Упрощение лямбда-выражений}. Иногда в процессе редукции возникает возможность заменить лямбда-выражение на его тело, что и делает данная трансформация. 

\textbf{Упрощение оператора when}. Оператор when, в случае не влияния на воспроизведение ошибки, может быть заменен на выражение в else ветке. Также необходимо обращать внимание на возможную вложенность операторов и начинать с оператора с наибольшей вложенностью.

\textbf{Упрощение блоков try-catch}. Содержимое данных блоков можно, в случае успешного воиспрозведения ошибки, поменять на тело блока try.


\section{Трансормации над текстовым представлением программы}
Трансформации над текстовым представлением программы появились из-за того, что некоторые трансформации ввиду сложности и скорости целесооборазно производить над текстовым представлением. Это может быть удаление какой-либо части текста, изменение какой-либо конструкции, подходящей под шаблон и т.д.

Первая трансформация из набора --- удаление текста внутри сбалансированной пары скобок. Скобки в языке программирования kotlin могут быть 4-х видов: фигурные, круглые, квадратные и треугольные. Данная трансформация находит все сбалансированные скобки в программе и пытается удалить весь текст внутри. В случае успешного воспроизведения ошибки с новым текстом программы данная трансормация применяется, в противном случае --- все возвращается обратно и производится переход к другой паре сбалансированных скобок. Следующая трансформация --- удаление скобок, поскольку часто в результате редукции появляются лишние скобки. Также в качестве дополнительной трансформации была придумана трансформация, удаляющая текст между каждой пары точек. Таким образом может быть удален лишний вызов функции. Например item.fun1.fun2 преобразуется в item.fun2

Следующая трансформация --- замена текста, подходящего под шаблон. Данная трансформация может быть трех видов:
\begin{itemize}
	\item замена текста, подходящего под шаблон, например "1294" на "0"; 
	\item замена части текста, подходящего под шаблон, например "i = i + 1" на "i++";
	\item применить к тексту, подходящему под шаблон другой шаблон, например "a + b + c + d" на "a + b".
\end{itemize}
Ниже приведен список разработанных шаблонов и их замен:
\begin{itemize}
	\item замена арифметических операторов += на =;
	\item замена while на if;
	\item замена всех констант на 0 и 1;
	\item замена строковых констант на "";
	\item замена всех типов на тип целого числа;
	\item упрощение бинарных операций.
\end{itemize}
TODO пример?


\section{Иерархический дельта-дебаггинг}
Иерархический дельта-дебаггинг является заключительной трансформацией и удаляет ту нерелевантную информацию, которую оставили предыдущие трансформации. Алгоритмы дельта-дебаггинга и иерархического дельта-дебаггинга приведены в главе 1. Данная работа работает для PSI представлением исходного кода. Поочередно к каждому уровню дерева, начиная с верхнего, применяется алгоритм классического дельта-дебаггинга, который находит минимальную конфигурацию на данном уровне и удаляет из дерева нерелевантные узлы. Данная трансформация может порождать синтаксически неверный код. Данный случай необходимо обрабатывать, так как проверка синтаксической корректности происходит намного быстрее, чем запуск теста и обработка сообщения об ошибке.

TODO пример??

