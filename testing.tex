\chapter{Тестирование системы}

В данном разделе производится исследование работоспособности разработанного прототипа на основе результатов его апробации на реальных проектах с ошибками компилятора языка Kotlin. Исследованию подвергаются два основных показателя: время работы и изменение размера файла, содержащего ошибку.
\section{Описание тестовых проектов}
Для апробации прототипа были выбраны следующие проекты, написанные на языке Kotlin:
\begin{itemize}
\item Результаты работы генератора случайных тестов для компилятора Kotlin: ~487 тестов, 198.7 Кб
\item kfg --- проект хуект. Размер файла с компиляторной ошибкой --- 40.2Кб. Размер проекта --- ~3500 строк  
\item kotoed
\item ktlint
\end{itemize}





\begin{table}[]
\center
\caption{\label{tab:ddminex2}Пример работы алгоритма дельта дебаггинга}
\begin{tabular}{| c | c | c | c | c | c | c |}
\hline
\bf \multirow{2}{*}{Проект} & \multicolumn{2}{|c|}{\bf Слайсинг} & \multicolumn{2}{|c|}{\bf Дельта дебаггинг}  & \multicolumn{2}{|c|}{\bf ReduKtor} \\
\cline{2-7}
& время & размер & время & размер & время & размер \\
\hline
Compiler tests & 2:30 & 65.6 & 32:23 & 33.9 & 31:55 & 21.6 \\
\hline
kotlinpoet & --- & --- & --- & --- & --- & --- \\
\hline
kfg & --- & --- & --- & --- & --- & --- \\
\hline
mapdb & --- & --- & --- & --- & --- & --- \\
\hline
kotoed & --- & --- & --- & --- & --- & --- \\
\hline
\end{tabular}
\end{table}

\section{Работа прототипа в тестовом режиме}
\section{Оценка целесообразности разработанного прототипа}
\section{Результаты тестирования}
\section{Анализ результатов}