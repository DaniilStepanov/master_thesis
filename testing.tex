\chapter{Тестирование системы}

В данном разделе производится исследование работоспособности разработанного прототипа на основе результатов его апробации на реальных проектах. В качестве тестируемой программы был выбран компилятор языка Kotlin. В качестве тестовой выборки были использованы результаты генератора случайных тестов для компилятора, а также несколько проектов на языке Kotlin, в которые исскуственно были внесены компиляторные ошибки. Исследованию подвергаются два основных показателя: время работы и изменение размера файла, содержащего ошибку. Редукция всех выбранных проектов запускалась в трех режимах: слайсинга, иерархического дельта-дебаггинга и с использованием всех реализованных трансформаций.
\section{Описание тестовых проектов}
Для апробации прототипа были выбраны следующие проекты, написанные на языке Kotlin:
\begin{itemize}
\item Результаты работы генератора случайных тестов для компилятора Kotlin: ~487 тестов, 198.7 Кб;
\item kotlinpoet --- программный интерфейс для генерации исходных файлов формата языка Kotlin. Размер файла с компиляторной ошибкой --- 15.2 кб. Размер проекта ~10 тыс. строк;
\item kfg --- проект для построения графа потока управления для Java байт-кода. Размер файла с компиляторной ошибкой --- 40.2Кб. Размер проекта --- ~3500 строк  ;
\item kotoed --- информационная система, автоматизирующая работу преподавателя при работе со студентами. Размер файла с комлпитяторной ошибкой ---. Размер проекта ---;
\item mapdb --- 2k
\item ktlint --- утилита для проверки стиля кода и его автоматического исправления.
\end{itemize}


\section{Оценка целесообразности}
Для оценки целесообразности технологии аппроксимации функ­ций разработанный прототип был дополнен кодом для измерения следующих показателей:
\begin{itemize}
	\item времени(t) работы прототипа в 3 режимах: слайсинга, иерархического дельта-дебаггинга и полного;
	\item размера(V) результирующего файла с ошибкой.
	
Использование прототипа является целесообразным при выполнении следующего соотношения: $t_{full} < t_{hdd}$ и $V_{full} << V_{slicing}$
\end{itemize}


\section{Результаты тестирования}
Результаты тестирования проектов приведены в таблице~\ref{tab:testing}.
\begin{table}[]
\center
\caption{\label{tab:testing}Результаты тестирования прототипа}
\begin{tabular}{| c | c | c | c | c | c | c |}
\hline
\bf \multirow{2}{*}{Проект} & \multicolumn{2}{|c|}{\bf Слайсинг} & \multicolumn{2}{|c|}{\bf Дельта дебаггинг}  & \multicolumn{2}{|c|}{\bf ReduKtor} \\
\cline{2-7}
& время & размер & время & размер & время & размер \\
\hline
Compiler tests & 2:30 & 65.6 & 32:23 & 33.9 & 31:55 & 21.6 \\
\hline
kotlinpoet & 2:35 & 15.1 & 80:10 & 1.8 & 16:15 & 0.3 \\
\hline
kfg & --- & --- & --- & --- & 18:48 & 0.04 \\
\hline
mapdb & --- & --- & 21:22 & 2.2 & 2:55 & 0.04 \\
\hline
kotoed & --- & --- & --- & --- & --- & --- \\
\hline
\end{tabular}
\end{table}
\section{Анализ результатов}