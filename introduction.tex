%%%%%%%%%%%%%%%%%%%%%%%%%%%%%%%%%%%%%%%%%%%%%%%%%%%%%%%%%%%%%%%%%%%%%%%%%%%%%%%%
\intro
%%%%%%%%%%%%%%%%%%%%%%%%%%%%%%%%%%%%%%%%%%%%%%%%%%%%%%%%%%%%%%%%%%%%%%%%%%%%%%%%
Тестирование --- неотъелемый этап совмеренного процесса разработки любого программного обеспечения. В настоящее время существует большое разнообразие методов выявления ошибок в программе, но всех их объединяет одна черта: часто тесты, приводящие к сбою программы, являются большими и содержат информацию, нерелевантную к ошибке. В таком случае программист должен определить, какая часть тестового примера вызывают отказ программы. Таким образом, в процессе тестирования появляется новый этап: локализация ошибки. 

В настоящее время локализация ошибки производится вручную, что очень часто приводит к непозволительно большим временным и трудозатратам. После чего, как правило, ошибка быстро устраняется. Автоматическое сокращение теста упрощает процесс отладки, поскольку в нем остается меньше нерелевантных деталей, что позволяет программисту сосредоточиться на причине возниконовения ошибки и способе ее исправления.

В данной работе представляется метод, автоматически локализующий найденную ошибку - программная редукция. Описанный метод разрабатывается для популярного~\cite{tiobe2018tiobe} языка программирования Kotlin. Программная редукция представляет собой комбинацию известных методов автоматической локализации ошибок и специфичных для языка Kotlin трансформаций. В рамках работы также планируется создание прототипа на основе предложенного подхода.

Работа состоит из пяти разделов. В разделе 1 рассматриваются известные методы автоматической локализации ошибок, а также производится их отбор и анализ. 

Раздел 2 посвящен постановке задачи редукции программ. Выбирается целевой язык программирования и обосновывается актуальность данной задачи для выбранного языка. В данном разделе также ставится задача разработки прототипа модуля редукции программ.

В разделе 3 описывается предлагаемый подход редукции программ. Рассматривается общая схема работы технологии, рассказывается об алгоритмах, применяемых в ней.


Разработка модуля описывается в разделе 4. В нем рассматри­ваются схема работы прототипа и особенности его реализации. Примеры исходного кода прототипа вынесены
в приложение 1.

В разделе 5 представляются результаты апробации разработан­ного прототипа на реальных программных проектах. Показывается эффективность и применимость разработанной технологии. Определяются недостатки прототипа и возможные доработки. 