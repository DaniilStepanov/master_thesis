%%%%%%%%%%%%%%%%%%%%%%%%%%%%%%%%%%%%%%%%%%%%%%%%%%%%%%%%%%%%%%%%%%%%%%%%%%%%%%%%
\intro
%%%%%%%%%%%%%%%%%%%%%%%%%%%%%%%%%%%%%%%%%%%%%%%%%%%%%%%%%%%%%%%%%%%%%%%%%%%%%%%%
Качество программного обеспечения жизненно важно для успеха любого современного программного проекта. В настоящее время существует большое разнообразие методов повышения качества программ (тестирование, инспекция, статический анализ и т.д.), но несмотря на это даже окончательные версии программных систем содержат множество ошибок. Часто тесты, приводящие к сбою программы, содержат большое количество информации, нерелевантной ошибке. В данном случае для поиска причины ошибки необходимо определять, какая часть тестового примера к ней приводит. Таким образом, в процессе тестирования появляется новый этап: локализация ошибки. 

В настоящее время локализация ошибки производится вручную, что очень часто приводит к непозволительно большим временным и трудовым затратам. Автоматическое сокращение теста упрощает процесс отладки, поскольку в нем остается меньше нерелевантных деталей, что позволяет программисту сосредоточиться на причине возникновения ошибки и способе ее исправления.

В данной работе представляется метод, автоматически локализующий найденную ошибку - метод программной редукции. Метод разрабатывается для популярного~\cite{tiobe2018tiobe} языка программирования Kotlin. Целью данной работы является разработка метода программной редукции путем комбинации известных методов автоматической локализации ошибок и специфичных для языка Kotlin программных трансформаций. В рамках работы также планируется создание прототипа на основе предложенного подхода.

Работа состоит из пяти разделов. В разделе 1 рассматриваются известные методы автоматической локализации ошибок, производится их сравнение, анализ и отбор. Также в разделе приведен обзор существующих инструментов автоматической редукции программ.

Раздел 2 посвящен постановке задачи редукции программ. Выбирается целевой язык программирования и обосновывается актуальность данной задачи для выбранного языка. В данном разделе также ставится задача разработки прототипа модуля редукции программ.

В разделе 3 описывается предлагаемый подход редукции программ. Рассматривается общая схема работы технологии, рассказывается об алгоритмах, применяемых в ней.


Разработка модуля описывается в разделе 4. В нем рассматри­ваются схема работы прототипа, его структура и особенности реализации. Примеры исходного кода прототипа вынесены в приложение 1.

В разделе 5 представляются результаты апробации разработан­ного прототипа на реальных программных проектах. Показывается эффективность и применимость разработанной технологии. Определяются недостатки прототипа и возможные доработки. 