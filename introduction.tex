%%%%%%%%%%%%%%%%%%%%%%%%%%%%%%%%%%%%%%%%%%%%%%%%%%%%%%%%%%%%%%%%%%%%%%%%%%%%%%%%
\intro
%%%%%%%%%%%%%%%%%%%%%%%%%%%%%%%%%%%%%%%%%%%%%%%%%%%%%%%%%%%%%%%%%%%%%%%%%%%%%%%%
В современном мире программное обеспечение используется во всех сферах человеческой жизни, поэтому проблема повышения его качества имеет очень большое значение. В настоящее время существует большое разнообразие методов повышения качества программ (тестирование, инспекция, статический анализ и т.д.), но несмотря на это даже окончательные версии программных систем содержат множество ошибок. Часто данные, приводящие к сбою программы, содержат большое количество информации, нерелевантной ошибке. В таком случае для поиска причины ошибки необходимо определять, какая часть входных данных к ней приводит. Таким образом, в процессе повышения качества программного обеспечения появляется новый этап: локализация найденных ошибок. 

В настоящее время локализация ошибок производится вручную, что очень часто приводит к непозволительно большим временным затратам. Автоматическое сокращение входных данных, приводящих к ошибке, упрощает процесс отладки, поскольку в них остается меньше нерелевантных деталей, что позволяет программисту лучше понять причину возникновения ошибки и способ ее исправления.

В данной работе представляется метод, автоматически локализующий найденную ошибку - метод программной редукции. Метод разрабатывается для популярного~\cite{tiobe2018tiobe} языка программирования Kotlin. Целью данной работы является разработка метода программной редукции путем комбинации известных методов автоматической локализации ошибок и специфичных для языка Kotlin программных трансформаций. В рамках работы также планируется создание прототипа на основе предложенного подхода.

Работа состоит из пяти разделов. В разделе 1 рассматриваются известные методы автоматической локализации ошибок, производится их сравнение, анализ и отбор. Также в разделе приведен обзор существующих инструментов автоматической редукции программ.

Раздел 2 посвящен постановке задачи редукции программ. Выбирается целевой язык программирования и обосновывается актуальность данной задачи для выбранного языка. В данном разделе также ставится задача разработки прототипа, реализующего технологию редукции программ.

В разделе 3 описывается предлагаемый подход редукции программ. Рассматривается общая схема работы технологии и рассказывается о применяемых в ней алгоритмах.

Разработка модуля описывается в разделе 4. В нем рассматри­ваются схема работы прототипа, его структура и особенности реализации. Примеры исходного кода прототипа вынесены в приложение~1.

В разделе 5 представляются результаты апробации разработан­ного прототипа на реальных программных проектах. Показывается эффективность и применимость разработанной технологии. Определяются недостатки прототипа и возможные его доработки. 