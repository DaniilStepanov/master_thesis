%%%%%%%%%%%%%%%%%%%%%%%%%%%%%%%%%%%%%%%%%%%%%%%%%%%%%%%%%%%%%%%%%%%%%%%%%%%%%%%%
\intro
%%%%%%%%%%%%%%%%%%%%%%%%%%%%%%%%%%%%%%%%%%%%%%%%%%%%%%%%%%%%%%%%%%%%%%%%%%%%%%%%
Тестирование --- неотъелемый этап совмеренного процесса разработки любого программного обеспечения. В настоящее время существует большое разнообразие методов выявления ошибок в программе, но всех их объединяет одна черта: часто тесты, приводящие к сбою программы, являются большими и содержат информацию, нерелевантную к ошибке. Таким образом, в процессе тестирования появляется новый этап: локализация ошибки, который в больших проектах может быть трудоемким и нести в себе большие временные затраты. 
НАЛИТЬ ВОДЫ!

В данной работе представляется метод, автоматически локализующий найденную ошибку - программная редукция. Описанный метод разрабатывается для популярного~\cite{tiobe2018tiobe} языка программирования Kotlin. Программная редукция представляет собой комбинацию известных методов автоматической локализации ошибок и специфичных для языка Kotlin трансформаций. В рамках работы также планируется создание прототипа на основе предложенного подхода.

Работа состоит из пяти разделов. В разделе 1 рассматриваются

Раздел 2 посвящен постановке задачи 

В разделе 3 описывается предлагаемый подход


Разработка модуля описывается в разделе 4. В нем рассматри­
ваются особенности реализации фреймворка Примеры исходного кода прототипа вынесены
в приложение 1.

В разделе 5 представляются результаты апробации разработан­ного прототипа на 