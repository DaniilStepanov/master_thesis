%%%%%%%%%%%%%%%%%%%%%%%%%%%%%%%%%%%%%%%%%%%%%%%%%%%%%%%%%%%%%%%%%%%%%%%%%%%%%%%%
\chapter{Пример применения разработанной технологии к компиляторным тестам, приводящим к его отказу}
%%%%%%%%%%%%%%%%%%%%%%%%%%%%%%%%%%%%%%%%%%%%%%%%%%%%%%%%%%%%%%%%%%%%%%%%%%%%%%%%

\begin{lstlisting}[caption = Исходный код компиляторного теста extensionFunction.kt376140234.kt]
class A() {

 fun String.test(OK: String) {

 }

}
fun box(): String {
val clazz = (A)?::class.java
val method = clazz.getDeclaredMethod("test", String::class.java, String::class.java)
val parameters = method.getParameters()
if ((! method[0].isImplicit() || parameters[0].isSynthetic())) {
return "wrong modifier on receiver parameter: ${parameters[0].modifiers}"
}
return parameters[1].name
}
\end{lstlisting}
\begin{lstlisting}[caption = Результат применения разработанного алгоритма редукции к файлу extensionFunction.kt376140234.kt]
fun box()  {
val clazz = (A)?::class.java
}
\end{lstlisting}

\begin{lstlisting}[caption = Исходный код компиляторного теста intReturnComplex4.kt1544248458.kt]
package test

public class Holder {

    public var value: String = ""

}

public inline fun doCall2_2(block: (() -> String), res: String, h: Holder): String {
    try {
        return doCall2_1(block, {
            h.value += ", OK_EXCEPTION"
            return "OK_EXCEPTION"
        }, res, h)
    } finally {
        h.value += ", DO_CALL_2_2_FINALLY"
    }
}

public inline fun <R> doCall2_1(block: (() -> R), exception: ((e: Exception) -> Unit), res: R, h: Holder): R {
    try {
        return doCall2<R>(block, exception, {
            try {
                h.value += ", OK_FINALLY"
                "OK_FINALLY"
            } finally {
                h.value += ", OK_FINALLY_NESTED"
            }
        }, res, h)
    } finally {
        h.value += ", DO_CALL_2_1_FINALLY"
    }
}

public inline fun <R> doCall2(block: (() -> R), exception: ((e: Exception) -> Unit), finallyBlock: (() -> Unit), res: R, h: Holder): R {
    try {
        try {
            return block()
        } catch (e: Exception) {
            (exception)!!(e)
        } finally {
            finallyBlock()
        }
    } finally {
        h.value += ", DO_CALL_2_FINALLY"
    }
    return res
}
\end{lstlisting}

\begin{lstlisting}[caption = Результат применения разработанного алгоритма редукции к файлу intReturnComplex4.kt1544248458.kt]
fun doCall2_2(res: String) {
    doCall2_1({}, res)
}

inline fun <R> doCall2_1(exception: (e: Exception) -> Unit, res: R) {
    doCall2<R>(exception)
}

inline fun <R> doCall2(exception: (e: Exception) -> Unit) {
    try {
    } catch (e: Exception) {
        (exception)!!
    }
}
\end{lstlisting}

\begin{lstlisting}[caption = Исходный код компиляторного теста kt242.kt-58744233.kt]
fun box(): String {
val i: (Int)? = 7
val j: (Int)? = null
val k = 7
if (i == 7)({})
if (7 == i)({})
if (j == 7)({})
if (7 == j)({})
if (i == k)({})
if (k == i)({})
if (j == k)({})
if (k == j)(((when {
(({}) ?: ({})) != null -> (({}) ?: ({}))
else -> (({}) ?: ({}))
})))
return "OK"
}
\end{lstlisting}

\begin{lstlisting}[caption = Результат применения разработанного алгоритма редукции к файлу kt242.kt-58744233.kt]
fun box()  {
(when {})
}
\end{lstlisting}

\begin{lstlisting}[caption = Исходный код компиляторного теста localClasses.kt-384517712inheritance.kt]
open class C(val grandParentProp: String)

fun box(): String {
    var sideEffects: String = ""
    var parentSideEffects: String = ""
    val justForUsageInClosure = 7
    val justForUsageInParentClosure = "parentCaptured"

    abstract class B : C {
        val B.parentProp: String

        init {
            sideEffects += "minus-one#"
            parentSideEffects += "1"
        }

        protected constructor(arg: Int) : super(justForUsageInParentClosure) {
            parentProp = (arg).toString()
            sideEffects += "0.5#"
            parentSideEffects += "#" + justForUsageInParentClosure
        }

        protected constructor(arg1: Int, arg2: Int) : super(justForUsageInParentClosure) {
            parentProp = (arg1 + arg2).toString()
            sideEffects += "0.7#"
            parentSideEffects += "#3"
        }

        init {
            sideEffects += "zero#"
            parentSideEffects += "#4"
        }

    }

    class A : B {
        var prop: String = ""

        init {
            sideEffects += prop + "first"
        }


        constructor(x1: Int, x2: Int) : super(x1, x2) {
            prop = x1.toString()
            sideEffects += "#third"
        }


        init {
            sideEffects += prop + "#second"
        }


        constructor(x: Int) : super(justForUsageInClosure + x) {
            prop += "${x}#int"
            sideEffects += "#fourth"
        }


        constructor() : this(justForUsageInClosure) {
            sideEffects += "#fifth"
        }


        override fun toString() = "$prop#$parentProp#$grandParentProp"
    }

    val a1 = A(5, 10).toString()
    if (a1 != "5#15#parentCaptured") {
        return "fail1: $a1"
    }
    if (sideEffects != "minus-one#zero#0.7#first#second#third") {
        return "fail2: ${sideEffects}"
    }
    if (parentSideEffects != "1#4#3") {
        return "fail3: ${parentSideEffects}"
    }
    sideEffects = ""
    parentSideEffects = ""
    val a2 = A(123).toString()
    if (a2 != "123#int#130#parentCaptured") {
        return "fail1: $a2"
    }
    if (sideEffects != "minus-one#zero#0.5#first#second#fourth") {
        return "fail4: ${sideEffects}"
    }
    if (parentSideEffects != "1#4#parentCaptured") {
        return "fail5: ${parentSideEffects}"
    }
    sideEffects = ""
    parentSideEffects = ""
    val a3 = A().toString()
    if (a3 != "7#int#14#parentCaptured") {
        return "fail6: $a3"
    }
    if (sideEffects != "minus-one#zero#0.5#first#second#fourth#fifth") {
        return "fail7: ${sideEffects}"
    }
    if (parentSideEffects != "1#4#parentCaptured") {
        return "fail8: ${parentSideEffects}"
    }
    return "OK"
}
\end{lstlisting}

\begin{lstlisting}[caption = Результат применения разработанного алгоритма редукции к файлу localClasses.kt-384517712inheritance.kt]
open class C(grandParentProp: String)

fun box() {
    val justForUsageInParentClosure = "parentCaptured"

    class B : C {
        val B.parentProp: String

        constructor(arg1: Int, arg2: Int) : super(justForUsageInParentClosure) {
            parentProp = arg1.toString()
        }
    }
}
\end{lstlisting}

\begin{lstlisting}[caption = Исходный код компиляторного теста safeCallWithElvis.kt-175330980.kt]
import kotlin.test.assertEquals
class A(val x : Int, val y : A?)
fun check(a: (A)?): Int {
    return a?.y?.x ?: (a?.x ?: 3)
}
fun checkLeftAssoc(a: (A)?): Int {
    ((return(a?.y?.x ?: (when {
        (a?.x) != null -> (a?.x)
        else -> (a?.x)
    }))) ?: (return(a?.y?.x ?: (when {
        (a?.x) != null -> (a?.x)
        else -> (a?.x)
    })))) ?: 3
}
fun box(): String {
    val a1 = A(2, A(1, null))
    val a2 = A(2, null)
    val a3 = null
    assertEquals(1, check(a1))
    assertEquals(2, check(a2))
    assertEquals(3, check(a3))
    assertEquals(1, checkLeftAssoc(a1))
    assertEquals(2, checkLeftAssoc(a2))
    assertEquals(3, checkLeftAssoc(a3))
    return "OK"
}
\end{lstlisting}

\begin{lstlisting}[caption = Результат применения разработанного алгоритма редукции к файлу safeCallWithElvis.kt-175330980.kt]
class A(val x: Int)

fun checkLeftAssoc(a: A?): Int {
    return (a?.x ?: (when {
        else -> a?.x
    }))
}
\end{lstlisting}